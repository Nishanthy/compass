\section*{Background}

It is often argued that the most effective visual analytics tools should support analysis at the rate of thought \cite{hanrahan:enthusiast}.
However, existing tools \cite{heer:dynamics} have not yet achieved this vision. \textit{Chart Typology} interfaces, such as a palette of chart templates found in spreadsheet software, are easy for view creation but difficult for view refinement.  \textit{Visualization Toolkits} \cite{bostock:d3, wilkinson:GoG} enable intricate designs but require coding, and therefore hinder rapid data exploration. Tableau \cite{stolte:polaris}, a state of the art visual analysis tool, enables rapid view exploration refinement as well as supports expressive visualizations creation.  However, creating an effective visualization in Tableau still requires both tool and design expertise.  To lower this barrier, Tableau's Show Me \cite{mackinlay:showme} feature automatically suggests suitable chart types for selected data attributes based on design practices \cite{tufte:visual, few:nowyouseeit}.  Nonetheless, it can suggest only a single visualization at a time although there are usually many appropriate visualizations.

A few research prototypes \cite{key:vizdeck, gotz:harvest, kandel:profiler} recommend a list of visualizations based on statistical properties of the data. While these systems may recommend relevant subsets of data, they produce only a limited range of visualizations and do not incorporate best practices of design.  Moreover, they lack interfaces for users to express their intention, leading to irrelevant suggestions.  An effective visualization recommender must address these limitations to successfully support users’ analysis.