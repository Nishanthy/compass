\section*{Recommendation Algorithm}

The recommendation process will involve \textit{generation}, \textit{ranking} and \textit{pruning}.  The generation algorithm will create \textit{relevant} visualization specifications,  which includes permutations of possible mappings between data attributes and visual variables, and combinations of marks types and data transformations that match user queries.  The system will then rank visualizations using a quality metric that we will develop and evaluate as part of this research. Finally, as top-ranked visualizations may contain redundant information,  we will use a similarity metric to prune the results to ensure diversity.

Our quality metric will combine multiple criteria including \textit{design quality} and \textit{interestingness}. The \textit{design quality} score will integrate design guidelines based on graphical perception studies \cite{cleveland:perception}.  For example, in perceptual effectiveness rankings of visual encodings of quantitative data \cite{mackinlay:apt}, length is more effective than angle.  Therefore, our algorithm will a priori prefer bar charts to pie charts.  To determine an \textit{interestingness} score, we will apply statistical measures such as mutual information\cite{wang:maximum} as well as anomaly and trend detection\cite{chandola:anomaly}. We will explore the use of interpretable machine learning techniques \cite{letham:interpretable} to combine our metrics, enabling our system to explain why a particular view is recommended. In the long term, we will extend the algorithm to learn from user interactions.