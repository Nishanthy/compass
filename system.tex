\section*{Scalable System Architecture}

Our visualization system will use a client-server infrastructure with a web-based client. The backend server will use database techniques to efficiently store and query data for the visualizations. Data can reside in the client, in the server's memory, or in persistent storage.  We will develop a query optimizer that automatically determines whether a particular computation should be computed on the client or on the server based on data size, data location, and latency.
Spare server resources will be used to find and profile anomalies and trends, expediting \textit{interestingness} score calculations.
Finally, we will explore big data techniques including index precomputation and data cubes to support fast brushing and linking \cite{liu:immens, lins:nanocubes}, data reduction methods such as sampling to simplify complex data, and online aggregation \cite{hellerstein:onlineagg, agarwal:blinkdb, fisher:trustme} to enhance responsiveness for larger data. We believe that coupling database systems and interactive visualizations will be performant and enable new functionalities \cite{wu:dvms}.
